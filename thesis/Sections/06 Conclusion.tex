\section{Conclusion}
\label{sec:Conclusion}
Pairs trading is a methodologically simple trading strategy: buy the stock that is undervalued and sell the one that is overvalued. The fair value of a stock is derived relative to another stock using past prices.
In this study, we examined the profitability of the famous distance method and the more sophisticated copula method. We find that both distance and copula methods are statistically significant profitable before transaction costs for S\&P 500 stocks from 2005 to 2022. Conversely, we find that the profitability of the copula method is driven by the large share of unconverged trades, which are slightly profitable. The small share of converged trades is not. To conclude whether the copula method does work or not additional research is required.

% sample size
Given the relatively small stock sample, we are not necessarily able to obtain the ideal pairs to nominate for trading. \citet{rad2016} for instance use the entire CRSP database. This results in 2,377 stocks and hence a total of 2,823,876 pairs being available for trading in the final period of their study. The distance method seems to have no problem with this since the returns found in this study are similar to the findings of \citet{rad2016, do2010}. However, the copula method might suffer from the pair quality caused by the small stock sample. To test this hypothesis, a study of the entire US equity market is required.

% Discussion of SSD
In this study, we use the SSD between normalised prices to identify pairs. This is in line with the pairs trading literature, though there might be better, more suitable pair identification methods for the copula method, especially in light of trading signals being derived using copulas.
Copulas provide a powerful way of decomposing the marginal and joint dependence structure. Using this already in the formation period, as \citet{krauss2017} did, can generate significant returns. The stock sample used tough is relatively small. Therefore researchers might want to apply their method to the entire US stock market as \citet{rad2016} did.



