\section{Introduction}

If stock markets are weakly efficient as described by \citet{fama1970}, then no statistically and economically significant excess returns should be obtainable by the information contained in past stock prices. However, it has been well documented that disarmingly simple trading strategies, so-called pairs trading strategies, can achieve statistically and economically significant excess returns using past stock prices \parencite{gatev2006,do2010,do2012,rad2016}. The question is whether these simple pairs trading strategies are still profitable and how more sophisticated strategies perform compared to them.

The idea of pairs trading is in theory simple: find a pair of stocks that share some pattern in prices, e.g., that did move closely together. If they diverge from this pattern, buy long the stock that is relatively undervalued and sell short the stock that is relatively overvalued. Once the stocks converge back to their historical pattern, i.e., equilibrium pricing, close the positions to realize a return. While in theory simple, the techniques to identify pairs in a first step, and generate trading signals in a second step can become quite sophisticated.

The most simple pairs trading strategy, the distance method, derives trading signals from the spread between normalised prices. If the spread is too large in absolute value, we simultaneously enter a long and short position and hope that the spread returns to zero to make a profit. A more sophisticated strategy is the copula method. 
Copulas enable us to model marginal and joint dependence structures independently of each other. This is useful, as equities are shown to exhibit asymmetry in their joint dependence \parencite{patton2004} and thus require a more flexible framework to model the dependence structure between two stocks \parencite{rad2016}.

In an extensive study, \citet{rad2016} analyse the profitability of pairs trading strategies on the entire US equity market from 1962 to 2014. In particular, they examine the distance, cointegration and copula method and find that all three strategies remain profitable even after transaction costs.
However, they also find that trading opportunities and profitability of the distance and cointegration method are declining, whereas the copula method is relatively stable in both dimensions.

The findings of \citet{rad2016} and the theoretical foundation of copulas make it worthwhile to further assess the suitability of copulas for pairs trading. In this study, we examine the profitability of the copula method on stocks that are part of the S\&P 500, a stock index which contains the largest US stocks in terms of market capitalisation. As a benchmark, we study the well-documented distance method. We find that, similar to \citet{rad2016} the distance and copula method generate statistically significant monthly excess returns of 0.57\% and 0.68\% before transaction costs for the top 20 pairs respectively. 
%While our findings for the distance method are in line with \citet{rad2016}, for the copula method we find that the profitability of this strategy largely stems from trades that did not converge, whereas the convergent trades produce negative excess returns. This suggests that the copula method is not able to model the marginal and joint dependence structure sufficiently well enough for the purpose of pairs trading. However, to make a final conclusion on the suitability of copula for pairs trading further research is required.

The remainder of this is study is structured as follows. Section \ref{sec:Literature Overview} provides an overview of recent pairs trading literature, with respect to the distance and copula method. Section \ref{sec:Copula Preliminaries} provides a formal introduction into the most important aspects of copulas in light of their application to pairs trading. Section \ref{sec:Pairs Trading Methodology} provides a detailed description of the distance and copula methodology. Section \ref{sec:Results} provides both the empirical findings of this study and a discussion. Section \ref{sec:Conclusion} concludes with potential issues that require further investigation of researchers. 



