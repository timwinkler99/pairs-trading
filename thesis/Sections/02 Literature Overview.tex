\section{Literature Overview}
\label{sec:Literature Overview}

\citet{gatev2006} assess the profitability of the simplest pairs trading strategy, namely the distance method, from 1962 until 2002 for the entire US stock market. They find that this strategy yields an average monthly excess return of 1.308\% for the distance method's top 5 pairs and 1.436\% for the distance method's top 20 pairs. Additionally, they find that pairs trading profitability was particularly strong in the 1970s and 1980s, however, this trend is declining since the 1990s. The authors suggest that this decline is due to an increase in arbitrage activity which aims at exploiting the profitability previously observed.
\citet{do2010,do2012} examine the robustness of the distance method's profitability with respect to commissions, market impact and short selling fees. They confirm the finding of \citet{gatev2006} of strong profitability during the 1970s and 1980s, but also the decline in profitability in subsequent years. Additionally, they find that pairs trading performs particularly well during the Dot-com (2000-2002) and Global Financial Crisis (2007-2009) periods. The authors conclude that on average, the baseline distance method as implemented in \citet{gatev2006} has lost its profitability, after controlling for costs and accounting for systematic risks. Pairs trading remains only slightly profitable for a few adjusted versions of the baseline distance method.

\citet{liew2013} not only analyse the distance method but also the copula method on a pre-selected, highly correlated pair of stocks from the health care sector. In particular, they make use of, what \citet{krauss2017} call, the return-based copula method, where conditional probabilities of relative mispricing are calculated every day during the trading period. If the conditional probabilities of relative mispricing are high/low enough a long short position is entered.
\citet{krauss2017} point out that this method comes with some flaws. First, pair selection is not copula based, it is based on a co-movement criterion. Second, the copula-based trading signals only depend on the most recent returns realized of both stocks in the pair. Thus, the time structure is lost. Besides these conceptional issues, there is no large-scale study of the return-based copula method which tests the robustness of this method.

\citet{xie2016} also apply the distance and copula method. They assess 89 stocks from the US utility sector from 2003 until 2012. Contrary to \citet{liew2013} their copula approach is, what \citet{krauss2017} call, the level-based copula method. Instead of using the daily obtained conditional probabilities in isolation, they construct relative mispricing indices, which are essentially the cumulative deviations from equilibrium pricing. They are thus able to retain the time structure of the pairs relative (mis-)pricing.
They find significant average monthly excess returns for the copula method which are higher than the distance method's returns.

\citet{rad2016} provide a recent and extensive study of pairs trading profitability. They analyse the entire US stock market from 1962 until 2014. In particular, they test the distance, cointegration and copula method. Their findings concerning the distance method are in line with \citet{gatev2006, do2010, do2012}. Similar to \citet{xie2016} they also make use of the level-based copula method, i.e., their trading decisions follow relative mispricing indices. They find that all three strategies are significantly profitable, even after accounting for transaction costs. The distance method is the most profitable strategy with an average monthly excess return of 0.91\% before transaction costs, followed by the cointegration and copula methods which have an average monthly excess return of 0.85\% and 0.43\% respectively.

The mentioned studies are all based on the pair selection process using the SSD between normalised prices during the formation period. \citet{krauss2017} suggest another procedure. 
Instead of using a co-movement measure, they extend the formation period, which is typically set to 12 months, to 60 months. During this 60 months formation period they use 12-month estimation periods, where Student-t copulas are fitted for all potential pair combinations. The estimation periods are followed by 1-month pseudo trading periods. They start a new estimation period every month, which results in 48 different portfolios during the formation period. After the 60-month formation period, they enter a 12-month trading period in which they trade the top $k$ pairs, $k\in\{5,10,20\}$, which are the ones that performed best during the entire formation period.
During the trading period, they distinguish between mean reversion and momentum pairs. The former refers to pairs which converge back to their equilibrium pricing level within the trading period. The latter consequently refers to pairs which do not converge back. The authors already  distinguish between these types of pairs during the trading process to apply different trading rules, in particular stop-loss rules, which aim at preventing high losses due to non-convergence.
They do this for stocks that are part of the S\&P 100 stock index from 1990 until 2014 and find that their approach is significantly profitable for the top $k$ mean-reversion and top $k$ momentum pairs. For the top 20 mean-reversion and momentum pairs, they find average monthly excess returns of 0.45\% and 0.55\% respectively.